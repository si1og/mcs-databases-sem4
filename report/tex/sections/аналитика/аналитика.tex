\section{Аналитика}

\subsection{Описание предметной области «Учёт компьютерных шрифтов в электронной типографии»}

% TODO: дополнить динамическим пересчётом (написать про него, в латехе, например, статический пересчёт)
% TODO: написать про проблему согласования печатной и предпечатной подготовки

Типографика~--- искусство оформления текста с помощью наборного шрифта~--- является одним из фундаментальных элементов визуальной коммуникации~--- передачи информации посредством зрительно воспринимаемых образов. На протяжении столетий она развивалась от ручного набора металлических литер (брусков с выпуклым изображением знака) до современных цифровых технологий. Изобретение печатного станка Иоганном Гутенбергом в середине пятнадцатого века положило начало массовому распространению печатного слова, однако процесс набора текста оставался трудоёмким ремеслом. Каждая литера отливалась из металла, а наборщик вручную составлял строки из отдельных знаков. Появление линотипа (машина, отливающая целые строки) и монотипа (машина, отливающая отдельные литеры) в конце девятнадцатого века механизировало этот процесс, но подлинная революция произошла с переходом к цифровым технологиям.

Электронная типография~--- организация, осуществляющая подготовку и выпуск печатной и цифровой продукции с использованием компьютерных технологий,~--- представляет собой современный этап развития печатного дела, в котором все процессы подготовки и воспроизведения текста осуществляются с использованием компьютерных технологий. В отличие от традиционной типографии, где шрифт существовал как физический объект, в электронной типографии шрифт является программным продуктом. Компьютерный шрифт~--- это файл, содержащий векторные или растровые описания графем (букв, цифр и других знаков письма), а также метрическую информацию, необходимую для корректного отображения и печати текста. Такие шрифты используются в издательских системах, графических редакторах, веб-страницах и пользовательских интерфейсах операционных систем.

В наши дни современную электронную типографию невозможно представить без обширных коллекций шрифтов. Дизайнеры, верстальщики и препресс-специалисты (специалисты по допечатной подготовке) ежедневно работают с десятками гарнитур, подбирая оптимальные решения для каждого проекта. Шрифтовая гарнитура объединяет набор начертаний, выполненных в едином стилистическом ключе. Начертания различаются насыщенностью штрихов (толщиной линий знака)~--- от сверхтонкого до сверхжирного~--- и наклоном знаков, формирующим прямое, курсивное (с изменённым рисунком букв) или наклонное (линейно наклонённое прямое) написание. Одна гарнитура может насчитывать от единственного начертания до нескольких десятков вариантов, что позволяет выделять структурные элементы документа~--- заголовки, подзаголовки, основной текст.

Шрифты классифицируются по конструктивным особенностям строения знаков. Антиквенные шрифты, также называемые серифными \textit{(от serif)}, имеют характерные засечки на концах штрихов. Гротески, или рубленые шрифты \textit{(sans serif fonts)}, лишены засечек и отличаются геометрической простотой форм. Моноширинные шрифты \textit{(monospace fonts)} характеризуются одинаковой шириной всех знаков, что делает их незаменимыми в программировании и табличной вёрстке. Акцидентные шрифты предназначены для заголовков, логотипов и декоративных целей, тогда как рукописные имитируют различные виды ручного письма. Шрифты также характеризуются стилистическими особенностями: геометрические построены на основе простых форм, гуманистические отличаются плавными формами, имитирующими ручное письмо пером, брусковые отличаются прямоугольными засечками и одинаковыми толщинами со штрихами.

Важной характеристикой компьютерного шрифта является поддержка языков и письменностей. Глифовый \textit{(от glyph,~графическое представление символа)} состав определяет, какие алфавиты способен отображать шрифт. Базовая латиница охватывает западноевропейские языки, расширенная латиница добавляет знаки для центральноевропейских и балтийских языков, кириллица обеспечивает поддержку славянских языков. Системы письма различаются также направлением чтения: латиница и кириллица читаются слева направо, арабское и еврейское письмо~--- справа налево, традиционные восточноазиатские тексты могут располагаться вертикально сверху вниз. Направление письма влияет на технические требования к отображению шрифта и должно учитываться при организации шрифтовой коллекции.

Компьютерные шрифты существуют в различных технических форматах. Формат TrueType, разработанный компанией Apple в конце восьмидесятых годов, получил широкое распространение благодаря поддержке со стороны операционной системы Windows. Формат OpenType, созданный совместными усилиями Microsoft и Adobe, расширил возможности шрифтов за счёт поддержки дополнительных типографических функций и большего количества глифов. Веб-шрифты в форматах WOFF и WOFF2 оптимизированы для передачи по сети и используются на веб-страницах. Отображение шрифта в цифровой среде требует постоянного пересчёта. При изменении размера кегля, ширины текстового блока или масштаба страницы программа заново вычисляет положение каждого символа, переносы слов, межстрочные и межбуквенные интервалы. Этот процесс происходит в текстовых редакторах, издательских системах и веб-браузерах. Чтобы сделать пересёт эффктивнее, существуют вариативные шрифты \textit{(variable fonts)}. Они расширяют возможности динамической адаптации, позволяя плавно изменять насыщенность, ширину и другие параметры начертания без загрузки отдельных файлов.

Электронная типография предполагает наличие парка рабочих станций, на которых выполняются задачи вёрстки, дизайна и допечатной подготовки. Рабочая станция представляет собой персональный компьютер с определёнными аппаратными характеристиками: процессором, объёмом оперативной памяти, типом и ёмкостью накопителя, параметрами графического адаптера. На каждой станции установлена операционная система~--- Windows, macOS или Linux~--- определённой версии. Сочетание аппаратных характеристик и операционной системы определяет, какие форматы шрифтов поддерживаются и насколько корректно они отображаются.

Шрифты в электронной типографии~--- лицензируемые продукты. Лицензия определяет допустимые способы использования: некоторые \newlineшрифты распространяются бесплатно для любых целей, другие требуют приобретения коммерческой лицензии, третьи доступны по подписке. Лицензионные условия могут ограничивать количество рабочих станций, на которых разрешена установка шрифта, тираж печатной продукции или число просмотров веб-страниц. Без учёта лицензий типография рискует нарушить авторские права.

В электронной типографии работают несколько категорий специалистов. Системный администратор отвечает за техническое состояние парка рабочих станций, установку операционных систем, настройку программного обеспечения и распределение шрифтовых ресурсов между станциями. Дизайнер осуществляет подбор шрифтов для проектов, формирует запросы на приобретение новых гарнитур и использует установленные шрифты в своей работе. Менеджер по закупкам ведёт учёт лицензий, оформляет приобретение новых шрифтов, контролирует сроки действия подписок и соответствие количества установок лицензионным ограничениям.

Процесс пополнения шрифтовой коллекции начинается с выявления потребности. Дизайнер, работая над проектом, может обнаружить необходимость в гарнитуре, отсутствующей в текущей коллекции типографии. Он формирует запрос, указывая название шрифта, требуемые начертания и обоснование выбора. Менеджер по закупкам проверяет наличие шрифта у издательств, сравнивает условия лицензирования и стоимость. После согласования бюджета оформляется приобретение лицензии. Системный администратор получает файлы шрифта и выполняет установку на рабочие станции, для которых предназначена лицензия.

Допечатная подготовка и печать требуют строгого соответствия шрифтов на всех этапах производства. Верстальщик создаёт макет на своей рабочей станции, используя определённые гарнитуры и начертания. Затем файл передаётся на станцию корректора, оттуда~--- оператору допечатной подготовки, который готовит материалы для вывода на печать. На каждом этапе программа обращается к шрифтам, установленным локально. Если на одной из станций нужный шрифт отсутствует, система подставит замену~--- визуально похожий шрифт с другими метриками. Это приведёт к смещению строк, изменению межбуквенных интервалов, нарушению переносов и выравнивания. Ошибка может остаться незамеченной до момента печати тиража. Исправление потребует перевёрстки, повторного согласования и дополнительных затрат. Централизованный учёт шрифтов позволяет контролировать идентичность коллекций на всех рабочих станциях и предотвращать подобные ситуации ещё на этапе передачи макета.

Процесс инвентаризации шрифтовых ресурсов направлен на поддержание актуальности сведений о коллекции. Системный администратор периодически проводит сверку фактически установленных шрифтов на рабочих станциях с данными учётной системы. Выявляются расхождения: шрифты, установленные без учёта, удалённые шрифты, изменения в конфигурации станций. По результатам инвентаризации учётные данные корректируются. Менеджер по закупкам проверяет соответствие количества установок лицензионным ограничениям и при необходимости приобретает дополнительные лицензии или деактивирует избыточные установки.

Таким образом, учёт компьютерных шрифтов в электронной типографии представляет собой комплексную задачу, охватывающую систематизацию сведений о шрифтовых гарнитурах и их характеристиках, регистрацию парка рабочих станций с их аппаратно-программными конфигурациями, отслеживание лицензионных условий и сроков их действия, а также фиксацию связей между конкретными шрифтами и станциями, на которых они установлены. Организация такого учёта обеспечивает соблюдение лицензионных требований, оперативный поиск необходимых шрифтов и рациональное использование типографических ресурсов.

\subsubsection{Цели функционирования базы данных}

\begin{enumerate}
    \item Хранение сведений о шрифтовых гарнитурах, их начертаниях и характеристиках.
    \item Хранение информации о рабочих станциях и их аппаратно-программных конфигурациях.
    \item Хранение информации о лицензиях на шрифты, условиях использования и сроках действия.
    \item Хранение информации об установленных шрифтах на каждой рабочей станции.
\end{enumerate}

\subsection{Описание сущностей}

\begin{enumerate}
    \item Гарнитура~--- это семейство начертаний шрифта, объединённых общим стилем и предназначенных для совместного использования в типографических работах.
    
    Атрибуты: название, год создания, описание, издательство, лицензия, число начертаний.

    \item Начертание~--- это конкретный вариант гарнитуры, определяемый насыщенностью штрихов и наклоном знаков, хранящийся в отдельном файле шрифта.
    
    Атрибуты: название, насыщенность, наклон, формат файла, размер файла.

    \item Категория~--- это классификационная группа, объединяющая шрифты по конструктивным особенностям строения знаков (например, по наличию засечек или ширине символов).
    
    Атрибуты: название, описание, иконка, число шрифтов.

    \item Формат~--- это технический стандарт хранения шрифтовых данных, определяющий структуру файла, способ описания знаков и совместимость с платформами.
    
    Атрибуты: название, расширение, год появления, описание, поддержка веб.

    \item Язык~--- это система письма с определённым алфавитом и направлением чтения, поддержка которой обеспечивается наличием соответствующих знаков в шрифте.
    
    Атрибуты: название, код ISO, направление письма, тип письменности.

    \item Издательство~--- это организация, занимающаяся разработкой, производством и распространением шрифтов, а также предоставлением лицензий на их использование.
    
    Атрибуты: название, страна, сайт, год основания, контактный email.

    \item Лицензия~--- это документ, устанавливающий права и ограничения на использование шрифта, включая количество рабочих станций и срок действия.
    
    Атрибуты: тип, срок действия, максимальное число станций, стоимость, дата приобретения.

    \item Рабочая станция~--- это персональный компьютер в составе парка типографии, используемый для выполнения задач вёрстки, дизайна и допечатной подготовки.
    
    Атрибуты: инвентарный номер, процессор, объём памяти, тип накопителя, операционная система.

    \item Операционная система~--- это комплекс программ, управляющий аппаратными ресурсами рабочей станции и обеспечивающий среду для работы прикладного программного обеспечения.
    
    Атрибуты: название, версия, разрядность, год выпуска.

    \item Установка~--- это факт размещения конкретного начертания шрифта на рабочей станции, фиксирующий связь между шрифтовым ресурсом и оборудованием.
    
    Атрибуты: дата установки, путь к файлу, статус, дата последнего использования.
\end{enumerate}

\subsection{Описание ролей и субъектов}

Системный администратор обслуживает парк рабочих станций. Он устанавливает операционные системы, настраивает программное обеспечение и распределяет шрифты между станциями.

Дизайнер использует шрифты в своей работе. Он подбирает гарнитуры для проектов и формирует запросы на приобретение новых шрифтов.

Менеджер по закупкам ведёт учёт лицензий. Он оформляет приобретение шрифтов, контролирует сроки действия подписок и соответствие количества установок лицензионным ограничениям.

Издательство выпускает шрифты и предоставляет лицензии на их использование. Типография приобретает лицензии у издательств и устанавливает шрифты на рабочие станции.